\chapter{Code Examples}

\begin{figure}[h]
\begin{lstlisting}[frame=tb]
type Node interface {
	Eval(*variables.Scope, *IOContainer) ExitStatus
}

type NodeIf struct {
	Condition Node
	Else      *NodeIf
	Body      Node
}

func (n NodeIf) Eval(scp *variables.Scope, ioc *IOContainer) ExitStatus {
	runBody := n.Condition.Eval(scp, ioc)
	if runBody == ExitSuccess {
		return n.Body.Eval(scp, ioc)
	}
	if n.Else != nil {
		return n.Else.Eval(scp, ioc)
	}
	return ExitSuccess
}     
\end{lstlisting}
\caption[The Node Interface]{The Node interface used in the AST. Also shows the NodeIf as an example of implementation. Interfaces are implicitly satisfied if the type has the required methods. \label{lst:node-interface}}
\end{figure}

\begin{figure}[h]
\begin{lstlisting}[frame=tb]
type NodeNoop struct{}

func (NodeNoop) Eval(*variables.Scope, *IOContainer) ExitStatus {
	return ExitSuccess 
}

type NodeEOF struct {
	NodeNoop
}
\end{lstlisting}
\caption[The EOF and No Op node]{The NodeEOF uses struct embedding to receive the methods of the NodeNoop.\label{lst:node-noop}}
\end{figure}


\begin{figure}[h]
\begin{lstlisting}[frame=tb]
func LeftShift(a, b int64) int64 {
	c := int64(math.Pow(2, float64(b)))
	if c == 0 {
		return a
	} else if c < 0 {
		panic("Negative Left Shift")
	}
	return a * c
}
func RightShift(a, b int64) int64 {
	c := int64(math.Pow(2, float64(b)))
	if c == 0 {
		return 0
	} else if c < 0 {
		panic("Negative Right Shift")
	}
	return a / c
}
\end{lstlisting}
\caption[Shift helper functions]{Go's compile time checks prevent using an int64 as an argument to a shift.\label{lst:arith-shift}}
\end{figure}

\begin{figure}[h]
\begin{lstlisting}[frame=tb]
switch currentCharacter:
case '<':
	switch l.next() {
	case '<':
		t = ArithLeftShift
		checkAssignmentOp = true
	case '=':
		t = ArithLessEqual
	default:
		t = ArithLessThan
		l.backup()
	}
}
if checkAssignmentOp {
	if l.hasNext('=') {
		t += ArithAssignDiff
	}
}
return t
\end{lstlisting}
\caption[Sample of Arithmetic Symbol Lexing]{Portion of the switch statement used to parse symbols. It shows how detection of the next token is possible using just 1 character of lookahead. '<', '<<', '<=' and '<<=' are all covered here.\label{lst:arith-symbol-lex}}
\end{figure}


\begin{figure}[h]
\begin{lstlisting}[frame=tb]
type ArithNode interface {
	nud(*Parser) int64
	led(int64, *Parser) int64
	lbp() int
}
\end{lstlisting}
\caption[ArithNode interface]{The interface required by the pratt parser for each node\cite{CROCKFORD-TDOP}. Differs from the examples by the addition of the parser arg to maintain state.
\label{lst:arith-node-interface}}
\end{figure}


\begin{figure}[h]
\begin{lstlisting}[frame=tb]
foofunc() {
	foo
}

true && {false; true} && echo "Reaches here"
echo "SUXXESS" | { tr X C && echo "Finished transform"} | tee /dev/null
\end{lstlisting}
\caption[Shell Command Grouping]{Shows command grouping being used in the definition of a function, a conditional list and in a pipe.
\label{lst:command-grouping}}
\end{figure}



\begin{figure}[h]
\lstinputlisting[language=c,caption={
[dash - evaltree]
The dispatch functions for evaluation of nodes in dash. Particularly confusing sections include the use of bit flags in the NAND/NOR/NSEMI case.
\label{lst:dash-evaltree}}]{Appendix3/evaltree.c}
\end{figure}



\begin{figure}[h]
\begin{lstlisting}[frame=tb]
if false; then
    $noop
else
    A="SUCCESS"
fi
\end{lstlisting}
\caption[Code for a simple if statement]{Shell code for an if statement. 'A' and 'noop' are previously set. The value of A is checked after the statement to ensure the code operated correctly.\label{lst:if-ast-1}}
\end{figure}

\begin{figure}[h]
\lstinputlisting[caption={
[AST for a simple if statement]
The AST produced for the code in~\ref{lst:if-ast-1}. 
\label{lst:if-ast-2}}]{Appendix3/AST-if.txt}
\end{figure}





