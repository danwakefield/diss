\chapter{Evaluation}

%Examiners expect to find in your dissertation a section addressing such questions as:

%\begin{itemize}
%   \item Were the requirements correctly identified? 
%   \item Were the design decisions correct?
%   \item Could a more suitable set of tools have been chosen?
%   \item How well did the software meet the needs of those who were expecting to use it?
%   \item How well were any other project aims achieved?
%   \item If you were starting again, what would you do differently?
%\end{itemize}

%Such material is regarded as an important part of the dissertation; it should demonstrate that you are capable not only of carrying out a piece of work but also of thinking critically about how you did it and how you might have done it better. This is seen as an important part of an honours degree. 

%There will be good things and room for improvement with any project. As you write this section, identify and discuss the parts of the work that went well and also consider ways in which the work could be improved. 

%Review the discussion on the Evaluation section from the lectures. A recording is available on Blackboard. 

\section{Requirements}
XXX


\section{Design Decisions}
As has been noted in~\ref{sec:implementation}, a few of the design decisions where clearly wrong. This section aims to show the problems and talk about possible future solutions to them.

\subsection{Lexer}
The biggest problem was the design of the lexer being non re-entrant.
This prevented many syntax features from being completed and has left the shell in a non compliant state.
This things that where affected include:
\begin{itemize*}
	\item Embedded variables in both the arithmetic construct and complex substitutions.
    \item Complex substitutions that take multiple arguments.
    \item Alias expansions.
\end{itemize*}

While substitutions and embedded variables could have been accomplished with sublexers this would have been quite inefficient. 
Each lexer currently requires a complete copy of the remaining input text, a problem when parsing large scripts.
Switching the lexers to take a Bytes.Buffer instead of a string for the input may help as slices\footnote{An array like structure that uses pointers into an actual array to reduce overhead.} could be taken from it.

Alias expansion is more complex as the input text can be modified considerably.
As they occur completely in the lexer they can introduce any text including multi-line strings or control structures.
This places strain on parts of the code that require knowledge of a tokens position in the source.
Error messages are particularly affected as if they indicate that problems occur on lines that don't exist or don't contain the error it would be very confusing. 

With a lightweight sublexer aliases could be expanded in a separate context and control returned to after the replacement in the original lexer.
Tokens could be modified to indicate how many alias replacements have occurred before they are emitted and the original line that the alias was expanded at.
This would require careful and complex bookkeeping however as aliases can be expanded multiple times.
Other shells also include the ability for expansions happen in places other that the first word of a command which would make it even more complex.

\subsection{Package Layout}
Another of the bad decisions was the lack of planning in package layout.

A few different workarounds where used as the timeframe would not have accommodated the work required to correct the problems.
The use of a placeholder package, for the ExitStatus and IOContainer types, and the map of empty interfaces for NodeFunctions are the main examples of problems caused by the lack of a decision in this area.

Every logical component should have been a package from the beginning which would have allowed real imports to work as intended.
It also would have allowed easy renaming of the variables package, using the builtin go fix tool.
This change has been needed since it became obvious that the package would end up holding more program state than just variables.

\subsection{Escaping}
Escaping is the act of turning certain sequences of characters into another character that is hard to input, represent or that has a special meaning.
The most well know example is '\verb!\n!' which is the ASCII char 92, backslash, followed by char 110, the letter 'n'.
This is converted into ASCII char 10, the newline.

The code behind this is not too difficult and some parts of it already exist as the lexer supports line continuation i.e. A backslash preceding a literal newline removes both from the input stream.
However this affects backslashes in all portions of the input text except raw strings.
A backslash can also be used to prevent alias expansion when it precedes a word rather than a character.



\section{Suitability of tools}
I am very happy with the decision to use Go as it has resulted in some extremely readable code in comparison to the C of bash and the C\verb!++! of zsh.




\section{Other project aims}

\section{What would you do differently.}













