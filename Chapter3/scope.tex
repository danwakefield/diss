\section{Scope}
\label{sec:scope}
The scope object was created during the second iteration alongside the arith parser.
It was also created as a subpackage because putting it in the main package was impossible since Go completely disallows circular references.

As is obvious from the package name, 'variables.Scope' and the headings in this section, the responsibilities of the class grew quickly and it quickly became mislabeled.
I realised the problems that had been created by the package layout
as I started to implement user functions in the penultimate iteration.
At this point it would have taken significant work to correct and the benefits would have been minimal.

\subsection{Variables}
The original Scope object was only used to get and set variables but it has lots of complexity.
It contains a list of hash tables which are searched in reverse order when setting, updating or retrieving anything.
The first value matching the name is used.

I created it this way as I knew that it would be needed by both commands with temporary assignments and functions using the local builtin.

Commands take a list of environment variables each as a string in the form, "VARNAME=FOO".
This meant that when I got to the stage when commands where being executed I had to add an Environ function to get them in the correct form.
The complexity of this function is O(2n), where n is the number of variables set in the shell, but I could not see a way to improve this without harming the performance of other aspects.

The design does not currently support variable flags.
This means that everything is exported to the environment of a new command.
Adding an export flag and a builtin function to set it would take little effort though.
Flags would also allow integer typed variables.
Changing the Variable struct to contain an isInteger bool and then when updating this variable use the result of 'arith.Parse(value)' instead of 'value'.

\subsection{User Functions}
\label{sec:user-functions}
User functions where added in the penultimate iteration.
The way they are stored in the Scope object shows how I was bitten by lack of forethought area of package layout.

\begin{verbatim}
type Scope struct {
	...
	Functions    map[string]interface{}
	...
}
\end{verbatim}

It shows that I have had to use a map of interface\{\} to store the NodeFunctions.
It is similar to a \verb!void *! in C as Go's implicit satisfaction of interfaces mean that everything satisfies the empty one.

This was done as storage and usage of a NodeFunction is done during the Eval phase which is defined in package main.
The solution would have been to create a subpackage for nodes.
However the work required to change the parser during the last week was greater than the bad practice of using interface\{\} here.

\subsection{Aliases}
Aliases where finished in the time frames.
They will however require storage in the Scope object and use of a builtin command to put them there.

The way the main lexer is created will also have to be different as aliases are expanded before anything is parsed but only when they are known about.

They also only take effect after after the compound block they are declared in is evaluated.
\begin{verbatim}
foofunc() {
    foo "IT WORKS"
}

alias foo=echo
foofunc
\end{verbatim}
This would cause the shell to print an error as the alias did not exist before the function was parsed.
foo will be available for the rest of the script but the foofunc function will always cause an error.
